\documentclass{anstrans}
%%%%%%%%%%%%%%%%%%%%%%%%%%%%%%%%%%%
\title{A Piecewise Quadratic Finite Element Basis with Application to Particle Transport}
\author{Michael W. Hackemack,$^{*}$ Jean C. Ragusa$^{*}$, Marvin L. Adams$^{*}$}

\institute{
$^{*}$Department of Nuclear Engineering, Texas A\&M Univeristy, 3133 TAMU, College Station, TX
}

\email{mike\_hack@tamu.edu \and jean.ragusa@tamu.edu \and mladams@tamu.edu}

% Optional disclaimer: remove this command to hide
%\disclaimer{Notice: This manuscript is a work of fiction. Any resemblance to actual articles, living or dead, is purely coincidental.}

%%%% packages and definitions (optional)
\usepackage{graphicx} % allows inclusion of graphics
\usepackage{booktabs} % nice rules (thick lines) for tables
\usepackage{microtype} % improves typography for PDF
%\usepackage{subfigure}
\usepackage{subcaption}
\usepackage{float}

\newcommand{\SN}{S$_N$}
\renewcommand{\vec}[1]{\bm{#1}} %vector is bold italic
\newcommand{\vd}{\bm{\cdot}} % slightly bold vector dot
\newcommand{\grad}{\vec{\nabla}} % gradient
\newcommand{\ud}{\mathop{}\!\mathrm{d}} % upright derivative symbol
\newcommand{\tcr}[1]{\textcolor{red}{#1}}

\newcommand{\Introfigpath}[1]{../../../Document/Rev0/figures/sec_Intro/{#1}}
\newcommand{\Snfigpath}[1]{../../../Document/Rev0/figures/sec_Sn/{#1}}
\newcommand{\BFfigpath}[1]{../../../Document/Rev0/figures/sec_BF/{#1}}

\begin{document}
%%%%%%%%%%%%%%%%%%%%%%%%%%%%%%%%%%%%%%%%%%%%%%%%%%%%%%%%%%%%%%%%%%%%%%%%%%%%%%%%
\section{Introduction}

%%%%%%%%%%%%%%%%%%%%%%%%%%%%%%%%%%%%%%%%%%%%%%%%%%%%%%%%%%%%%%%%%%%%%%%%%%%%%%%%
\section{Polygonal Basis Functions}

%--------------------------------------------
\subsection{Piecewise Linear (PWL) Basis Functions}
%--------------------------------------------



%--------------------------------------------
\subsection{Piecewise Quadratic (PWQ) Basis Functions}
%--------------------------------------------

% serendipity overview figure
% ---------
\begin{figure*}[hbt]
\centering
\includegraphics[width=0.90\textwidth]{\BFfigpath{linear_to_quad_process.png}}
\caption{Overview of the process to construct the quadratic serendipity basis functions on polygons.}
\label{fig::2D_quad_process}
\end{figure*}
% ---------

%%%%%%%%%%%%%%%%%%%%%%%%%%%%%%%%%%%%%%%%%%%%%%%%%%%%%%%%%%%%%%%%%%%%%%%%%%%%%%%%
\section{Results and Analysis}


%%%%%%%%%%%%%%%%%%%%%%%%%%%%%%%%%%%%%%%%%%%%%%%%%%%%%%%%%%%%%%%%%%%%%%%%%%%%%%%%
\newpage
\section{Conclusions}


%%%%%%%%%%%%%%%%%%%%%%%%%%%%%%%%%%%%%%%%%%%%%%%%%%%%%%%%%%%%%%%%%%%%%%%%%%%%%%%%
%\appendix
%\section{Appendix}

%%%%%%%%%%%%%%%%%%%%%%%%%%%%%%%%%%%%%%%%%%%%%%%%%%%%%%%%%%%%%%%%%%%%%%%%%%%%%%%%
%\section{Acknowledgments}
%This material is based upon work supported by the Department of Energy Rickover Fellowship Program in Nuclear Engineering.

%%%%%%%%%%%%%%%%%%%%%%%%%%%%%%%%%%%%%%%%%%%%%%%%%%%%%%%%%%%%%%%%%%%%%%%%%%%%%%%%
\bibliographystyle{ans}
\bibliography{references}

\end{document}

