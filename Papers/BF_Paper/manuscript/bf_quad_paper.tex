%%
%% This is file `elsarticle-template-num.tex',
%% generated with the docstrip utility.
%%
%% The original source files were:
%%
%% elsarticle.dtx  (with options: `numtemplate')
%% 
%% Copyright 2007, 2008 Elsevier Ltd.
%% 
%% This file is part of the 'Elsarticle Bundle'.
%% -------------------------------------------
%% 
%% It may be distributed under the conditions of the LaTeX Project Public
%% License, either version 1.2 of this license or (at your option) any
%% later version.  The latest version of this license is in
%%    http://www.latex-project.org/lppl.txt
%% and version 1.2 or later is part of all distributions of LaTeX
%% version 1999/12/01 or later.
%% 
%% The list of all files belonging to the 'Elsarticle Bundle' is
%% given in the file `manifest.txt'.
%% 

%% Template article for Elsevier's document class `elsarticle'
%% with numbered style bibliographic references
%% SP 2008/03/01

%\documentclass[preprint,12pt]{elsarticle}
\documentclass[preprint,10pt]{elsarticle}
%\documentclass[final,3p,times]{elsarticle} 

%% Use the option review to obtain double line spacing
%% \documentclass[authoryear,preprint,review,12pt]{elsarticle}

%% Use the options 1p,twocolumn; 3p; 3p,twocolumn; 5p; or 5p,twocolumn
%% for a journal layout:
%% \documentclass[final,1p,times]{elsarticle}
%% \documentclass[final,1p,times,twocolumn]{elsarticle}
%% \documentclass[final,3p,times]{elsarticle}
%% \documentclass[final,3p,times,twocolumn]{elsarticle}
%% \documentclass[final,5p,times]{elsarticle}
%% \documentclass[final,5p,times,twocolumn]{elsarticle}

%% if you use PostScript figures in your article
%% use the graphics package for simple commands
%\usepackage{subfigure}
\usepackage{caption}
\usepackage{subcaption}

\usepackage{color}
%% or use the graphicx package for more complicated commands
\usepackage{graphicx}
%% or use the epsfig package if you prefer to use the old commands
\usepackage{epstopdf}

%% The amssymb package provides various useful mathematical symbols 
%% The amsthm package provides extended theorem environments
\usepackage{amssymb}
\usepackage{amsmath}
% more math
\usepackage{amsfonts}
\usepackage{amstext}
\usepackage{amsbsy}
%\usepackage{mathbbol} 

%% The lineno packages adds line numbers. Start line numbering with
%% \begin{linenumbers}, end it with \end{linenumbers}. Or switch it on
%% for the whole article with \linenumbers.
\usepackage{lineno}

\journal{Journal of Comp. Phys.}
%%%%%%%%%%%%%%%%%%%%%%%%%%%%%%%%%%%%%%%%%%%%%%%%%%%%%%%%%%%%%%%%%%%%
% operators
\renewcommand{\div}{\vec{\nabla}\! \cdot \!}
\newcommand{\grad}{\vec{\nabla}}
% latex shortcuts
\newcommand{\bea}{\begin{eqnarray}}
\newcommand{\eea}{\end{eqnarray}}
\newcommand{\be}{\begin{equation}}
\newcommand{\ee}{\end{equation}}
\newcommand{\bal}{\begin{align}}
\newcommand{\eali}{\end{align}}
\newcommand{\bi}{\begin{itemize}}
\newcommand{\ei}{\end{itemize}}
\newcommand{\ben}{\begin{enumerate}}
\newcommand{\een}{\end{enumerate}}
% DGFEM commands
\newcommand{\jmp}[1]{[\![#1]\!]}                     % jump
\newcommand{\mvl}[1]{\{\!\!\{#1\}\!\!\}}             % mean value
\newcommand{\keff}{\ensuremath{k_{\textit{eff}}}\xspace}
% shortcut for domain notation
\newcommand{\D}{\mathcal{D}}
% vector shortcuts
\newcommand{\vo}{\vec{\Omega}}
\newcommand{\vr}{\vec{r}}
\newcommand{\vn}{\vec{n}}
\newcommand{\vnk}{\vec{\mathbf{n}}}
\newcommand{\vj}{\vec{J}}
\newcommand{\eig}[1]{\| #1 \|_2}

\newcommand{\EI}{\mathcal{E}_h^i}
\newcommand{\ED}{\mathcal{E}_h^{\partial \D^d}}
\newcommand{\EN}{\mathcal{E}_h^{\partial \D^n}}
\newcommand{\ER}{\mathcal{E}_h^{\partial \D^r}}
\newcommand{\reg}{\textit{reg}}

% extra space
\newcommand{\qq}{\quad\quad}
% common reference commands
\newcommand{\eqt}[1]{Eq.~(\ref{#1})}                     % equation
\newcommand{\fig}[1]{Fig.~\ref{#1}}                      % figure
\newcommand{\tbl}[1]{Table~\ref{#1}}                     % table
\newcommand{\sct}[1]{Section~\ref{#1}}                   % section

\newcommand\br{\mathbf{r}}
%\newcommand{\tf}{\varphi}
\newcommand{\tf}{b}

\newcommand{\tcr}[1]{\textcolor{red}{#1}}
\newcommand{\mt}[1]{\marginpar{ {\tiny #1}}}
\newcommand{\Introfigpath}[1]{../../../Document/Rev0/figures/sec_Intro/{#1}}
\newcommand{\Snfigpath}[1]{../../../Document/Rev0/figures/sec_Sn/{#1}}
\newcommand{\BFfigpath}[1]{../../../Document/Rev0/figures/sec_BF/{#1}}
%%%%%%%%%%%%%%%%%%%%%%%%%%%%%%%%%%%%%%%%%%%%%%%%%%%%%%%%%%%%%%%%%%%%%
%
%   BEGIN DOCUMENT
%
%%%%%%%%%%%%%%%%%%%%%%%%%%%%%%%%%%%%%%%%%%%%%%%%%%%%%%%%%%%%%%%%%%%%%
\begin{document}

 

%%%%%%%%%%%%%%%%%%%%%%%%%%%%%%%%%%%%%%%%%%%%%%%%%%%%%%%%%%%%%%%%%%%%
\begin{frontmatter}

%% Title, authors and addresses

%% use the tnoteref command within \title for footnotes;
%% use the tnotetext command for theassociated footnote;
%% use the fnref command within \author or \address for footnotes;
%% use the fntext command for theassociated footnote;
%% use the corref command within \author for corresponding author footnotes;
%% use the cortext command for theassociated footnote;
%% use the ead command for the email address,
%% and the form \ead[url] for the home page:
%\title{Title\tnoteref{label1}}
%% \tnotetext[label1]{}
%% \author{Name\corref{cor1}\fnref{label2}}
%% \ead{email address}
%% \ead[url]{home page}
%% \fntext[label2]{}
%% \cortext[cor1]{}
%% \address{Address\fnref{label3}}
%% \fntext[label3]{}

%-------------------------
%-------------------------
%\title{Discontinuous Finite Element Solution for Diffusion Equations on Arbitrary Polygonal Meshes}
%\title{Higher-Order Basis Functions for the DGFEM $S_N$ Transport Equation on Arbitrary Polygonal Grids}
\title{Higher-Order Discontinuous Finite Element Discretization for $S_N$ Transport on Arbitrary Polygonal Grids}
%-------------------------
%-------------------------

%% use optional labels to link authors explicitly to addresses:
%% \author[label1,label2]{}
%% \address[label1]{}
%% \address[label2]{}

%-------------------------
\author{Michael W. Hackemack}
\author{Jean C. Ragusa}
\ead{jean.ragusa@tamu.edu}
\address{Department of Nuclear Engineering, Texas A\&M University, College Station, TX 77843, USA}

% \cortext[cor1]{Corresponding author}
%-------------------------

%-------------------------
\begin{abstract}

abstract goes here...
 

\end{abstract}
%-------------------------

%-------------------------
\begin{keyword}
  Radiation Transport \sep
	Arbitrary Polygonal Grids \sep
  Discontinuous Finite Element \sep
	Higher-Order Basis Functions
\end{keyword}
%-------------------------

\end{frontmatter}

%%%%%%%%%%%%%%%%%%%%%%%%%%%%%%%%%%%%%%%%%%%%%%%%%%%%%%%%%%%%%%%%%%%%

\linenumbers

%%%%%%%%%%%%%%%%%%%%%%%%%%%%%%%%%%%%%%%%%%%%%%%%%%%%%%%%%%%%%%%%%%%%
%%%%%%%%%%%%%%%%%%%%%%%%%%%%%%%%%%%%%%%%%%%%%%%%%%%%%%%%%%%%%%%%%%%%
\section{Introduction} \label{sec::intro}
%%%%%%%%%%%%%%%%%%%%%%%%%%%%%%%%%%%%%%%%%%%%%%%%%%%%%%%%%%%%%%%%%%%%
%%%%%%%%%%%%%%%%%%%%%%%%%%%%%%%%%%%%%%%%%%%%%%%%%%%%%%%%%%%%%%%%%%%%
%\begin{itemize}
%\item 
%\item 
%\item 
%\end{itemize}

\tcr
{
\begin{enumerate}
\item introduce continuous, 1G transport equation
\begin{itemize}
\item application of TE (reactors, HDELP, astrophy, medical,...)
\item oin this paper, we focus on solution techniques based on first-order form of the TE, namely, Sn DO in angle and DFEM in space. Why? The spatial disc of Sn transport dates back to Lessain/Raviart 1972, Reed/Hill 1973, and amenable to efficient iterative techniques, matrix-free transport sweeps + DSA precondition. Works well on massively // architectures.  
\item most of DGFEM Sn Transport code (Denovo, Partisn, Capsacin) use linear on simplices and hex grids, except PDT, PWLD Adams on abitrary polyg/h
\item also bring some focus on TDL (thick)
\end{itemize}
\item history of higher-order Sn transport discretizations
\begin{itemize}
\item DGFEM: Reed high-order triangular, Wang (AMR), 
\item nodal method transport (review paper), Azmy AHOT, Wachspress 
\item acknowledge basis functions derived/used here can be employed for other spatial disc of transport on arbitrary grids, SAAF, LS, EP, (CFEM) [some form can be Sn in agle or Pn, e.g., SAAF-PN/SN, EVENT UK Imperial  + RATTLESNAKE). Important to recall this in conclusions. 
\end{itemize}
\item polygonal grids + poly math
\begin{itemize}
\item HOWEVER, all of the above: nada for polygons except PWLD which is only linear
\item why polyg? 4 reasons 
\item Rand/Gillete sukumar
\end{itemize}
\item outline
\begin{itemize}
\item brief review DG/ Sn
\item brief review of linear basis on polygon
\item the meat: (1) application (new) of existing HO basis function to Transport (2) novelty: PWQ on poly
\item AMR
\item results
\end{itemize}
\end{enumerate}
}

We can summarize the benefits of using polygonal meshes as the following:

\begin{enumerate}
	\item Polygonal mesh cells are now being employed in other physics communities - notably computational fluid dynamics (CFD) \cite{ref::star_CCM} and solid mechanics \cite{yip2005automated};
	\item They are believed to reduce the number of unknowns to solve with equivalent accuracy;
	\item They can reduce cell/face counts which can reduce algorithm wallclock times depending on the solution method;
	\item They can allow for transition elements between different portions of the domain (e.g., triangular elements bordering quadrilateral elements at the border of the boundary layer);
	\item They can easily be split along cut planes - allowing the mesh to be partitioned into regular or irregular divisions as well as be generated by simplical meshing techniques across processor sets in parallel;
	\item Hanging nodes from non-conforming meshes, like those that naturally arise from locally refined/adapted meshes as seen in Figure \ref{fig::locally_refined_vertices}, are not necessary. 
\end{enumerate}

% Refined quadrilateral vertices figure
% ---------
\begin{figure}[hbt]
\centering
\includegraphics[width=0.85\textwidth]{\Introfigpath{locally_refined_vertices.png}}
\caption{Local mesh refinement of an initial quadrilateral cell (left) leads to a degenerate pentagonal cell (right) without the use of a hanging node.}
\label{fig::locally_refined_vertices}
\end{figure}
% ---------

% Outline 
% --------
The remainder of the paper is as follows.

%%%%%%%%%%%%%%%%%%%%%%%%%%%%%%%%%%%%%%%%%%%%%%%%%%%%%%%%%%%%%%%%%%%%
%%%%%%%%%%%%%%%%%%%%%%%%%%%%%%%%%%%%%%%%%%%%%%%%%%%%%%%%%%%%%%%%%%%%
\section{The DGFEM $S_N$ Discretization of the Transport Equation} \label{sec::dgfem}
%%%%%%%%%%%%%%%%%%%%%%%%%%%%%%%%%%%%%%%%%%%%%%%%%%%%%%%%%%%%%%%%%%%%
%%%%%%%%%%%%%%%%%%%%%%%%%%%%%%%%%%%%%%%%%%%%%%%%%%%%%%%%%%%%%%%%%%%%
\tcr
{ keep it short
\begin{enumerate}
\item angular discretization
\item DGFEM discretization with upwind scheme
\item theoretical convergence rates (including solution irregularity) Wang + others (Pitkarantka ...)
\end{enumerate}
}

In this section, we review the DGFEM $S_N$ transport equation. Given an angular quadrature set, $\{  \vec{\Omega}_m, w_m \}_{m=1}^M$, the $S_N$ transport equation with isotropic scattering is defined within an open, convex spatial domain $\mathcal{D}$, with boundary, $\partial \mathcal{D}$, as

\begin{equation}
\label{eq::trans_eq_simple_no_energy_groups}
\left( \vec{\Omega}_m \cdot \vec{\nabla}  + \sigma_{t} (\vec{r}) \right)  \psi_{m} (\vec{r})=  \frac{\sigma_{s} (\vec{r})}{4 \pi}  \phi (\vec{r}) + Q_{m} (\vec{r}) ,
\end{equation}

\noindent where $\psi_{m} (\vec{r}) = \psi (\vec{r}, \vec{\Omega}_m)$ is the angular flux at position $\vec{r}$ in direction $\vec{\Omega}_m$, $\sigma_{t}$ and $\sigma_{s}$ are the total and scattering cross sections, and $Q_{m}$ is a distributed source for angle $m$. The scalar flux is defined as

\begin{equation}
\label{eq::scalar_flux_int}
\phi (\vec{r}) = \int\limits_{4 \pi} d \Omega \, \psi (\vec{r}, \vec{\Omega}) \approx \sum_{m=1}^{M} w_m \psi_{m} (\vec{r}) .
\end{equation}

\noindent We then lay down an unstructured mesh, $\mathbb{T}_h \in \mathbb{R}^{d}$, over the spatial domain, where $d$ is the dimensionality of the problem (for this work, $d=2$). This mesh consists of non-overlapping spatial elements to form a complete union over the entire spatial domain: $\mathcal{D} = \bigcup_{K \in \mathbb{T}_h} K$. 

We next form the set of DGFEM equations for a polygonal element $K$ with $n$ vertices and $n$ faces. We multiply Eq. (\ref{eq::trans_eq_simple_no_energy_groups}) by an appropriate test function $b_m$, integrate over element $K$, and apply Gauss' Divergence Theorem to the streaming term to obtain the Galerkin weighted-residual for element $K$ for an angular direction $\vec{\Omega}_m$:

\begin{equation}
\label{eq::DGFEM_trans_eq_cellK}
\begin{aligned}
 \Big< ( \vec{\Omega}_m \cdot \vec{n} ) \, b_m, \tilde{\psi}_m  \Big>_{\partial K} - \left( \vec{\Omega}_m \cdot  \vec{\nabla} b_m, \psi_{m} \right)_{K}  + \Big(  \sigma_{t} b_m ,   \psi_{m} \Big)_{K} \\
=  \left( \frac{\sigma_s}{4 \pi} b_m ,   \phi \right)_{K} + \left(  b_m ,   Q_{m} \right)_{K}
\end{aligned} .
\end{equation}

The inner products,

\begin{equation}
\label{eq::spatial_inner_products_cell}
 \Big( u, v \Big)_K \equiv \int_K u \, v \, d r \qquad \text{and} \qquad  \Big< u, v \Big>_{\partial K} \equiv \int_{\partial K} u \, v \, d s
\end{equation} 

\noindent correspond to integrations over the cell volume and faces, respectively, where $dr \in \mathbb{R}^d$ is within the cell and $ds \in \mathbb{R}^{d-1}$ is along the cell boundary.

The spatial convergence of DGFEM methods for the hyperbolic systems has been extensively studied \cite{lesaint1974finite,houston2000stabilized,houston2002discontinuous,wang2009convergence}. With the discretized flux solution, $\phi_h \in W^h_{\mathcal{D}}$, corresponding to our unstructured, $\mathbb{T}_h$, we can define an error for our DGFEM transport solution with the $L_2$ norm as

\begin{equation}
\label{eq::Convergence_L2norm_phi}
|| \phi_h - \phi_{exact} ||_{L_2} \leq C \frac{h^{q}}{(p+1)^{q-1/2}} .
\end{equation}

\noindent In Eq. (\ref{eq::Convergence_L2norm_phi}), $q = \min (p+1, r )$, $h$ is the maximum diameter of all the mesh elements, $p$ is the polynomial completeness of the finite element function space, $r$ is the regularity index of the transport solution, and $C$ is a constant that is independent of the mesh employed.

%%%%%%%%%%%%%%%%%%%%%%%%%%%%%%%%%%%%%%%%%%%%%%%%%%%%%%%%%%%%%%%%%%%%
%%%%%%%%%%%%%%%%%%%%%%%%%%%%%%%%%%%%%%%%%%%%%%%%%%%%%%%%%%%%%%%%%%%%
\section{Linear Polygonal Basis Functions} \label{sec::linpoly}
%%%%%%%%%%%%%%%%%%%%%%%%%%%%%%%%%%%%%%%%%%%%%%%%%%%%%%%%%%%%%%%%%%%%
%%%%%%%%%%%%%%%%%%%%%%%%%%%%%%%%%%%%%%%%%%%%%%%%%%%%%%%%%%%%%%%%%%%%
\tcr
{
\begin{enumerate}
\item Image of geometric terms on general polygon
\item List/description of barycentric properties
\end{enumerate}
}

\begin{figure}
\centering
\includegraphics[width=0.65\textwidth]{\BFfigpath{ref_polygon_Rev1.png}}
\caption{Arbitrary polygon with geometric properties used for 2D basis function generation.}
\label{fig::2D_ref_polygon}
\end{figure}

\begin{enumerate}
\item Positivity: $\lambda_j \geq 0$;
\item Partitiion of unity: $\sum\limits_{j=1}^n \lambda_j = 1$;
\item Affine combination: $\sum\limits_{j=1}^n \vec{r}_j \lambda_j (\vec{r}) = \vec{r}$;
\item Lagrange property: $\lambda_i (\vec{r}_j) = \delta_{ij}$;
\item Piecewise boundary linearity: $\lambda_j ((1-\mu ) \vec{x}_j  + \mu \vec{x}_{j+1})  = (1-\mu ) \lambda_j (\vec{x}_j ) + \mu \lambda_j (\vec{x}_{j+1} )$
\end{enumerate}

%------------------------------------------------------------------------------------------------------------
\subsection{Wachspress Rational Functions}
%------------------------------------------------------------------------------------------------------------

The first linearly-complete polygonal functions that we will consider are the Wachspress rational functions \cite{wachspress1975rational}. These rational functions were the first derived for 2D polygons and possess all the properties of the barycentric coordinates. However, they are only valid interpolants over strictly-convex polygons. Weakly-convex polygons (contain colinear vertices) lose piecewise linearity on the boundary. Concave polygons have regions within their domain that are undefined and divide-by-zero occurs. The Wachspress coordinates (which we denote as $\lambda^W$) have the following form

\begin{equation}
\label{eq::wach_BF}
\lambda_{j}^{W} (\vec{r}) = \frac{w_j (\vec{r}) }{\sum\limits_{i=1}^{n} w_i (\vec{r})},
\end{equation}

\noindent where the Wachspress weight function for vertex $j$, $w_j$, has the following definition:

\begin{equation}
\label{eq::wach_weights}
w_j (\vec{r})  = \frac{A(\vec{r}_{j-1}, \vec{r}_{j}, \vec{r}_{j+1})}{A(\vec{r}, \vec{r}_{j-1}, \vec{r}_{j}) \, A(\vec{r}, \vec{r}_{j}, \vec{r}_{j+1})} .
\end{equation}

\noindent In Eq. (\ref{eq::wach_weights}), the terms $A(\vec{a}, \vec{b}, \vec{c})$ denote the signed area of the triangle with vertices $\vec{a}$, $\vec{b}$, and $\vec{c}$. Each of these signed areas can be computed by

\begin{equation}
\label{eq::wach_signed_area}
A(\vec{a}, \vec{b}, \vec{c}) = \frac{1}{2}
\left|  
  \begin{array}{ccc}
  1 & 1 & 1 \\
  x_a & x_b & x_c \\
  y_a & y_b & y_c
  \end{array}
\right| .
\end{equation}

\noindent Through observation of Eq. (\ref{eq::wach_weights}), we can see that the denominator of $w_j$ goes to zero on 

%------------------------------------------------------------------------------------------------------------
\subsection{Piecewise Linear Functions}
%------------------------------------------------------------------------------------------------------------

The second linearly-complete 2D polygonal functions that we will analyze are the Piecewise Linear (PWL) functions proposed by Stone and Adams \cite{ref::PWLD_stone_adams,ref::PWLD_stone_adams_unstructured}. The 2D PWL functions are defined as combinations of linear triangular functions, with some of them only having measure within a subregion of a polygon. These subregions are formed by triangulating the arbitrary 2D polygon into a set of sub-triangles.  Each sub-triangle is defined by two adjacent vertices of the polygon and the polygon's centroid, $\vec{r}_{c}$. The polygon's centroid can be defined by

\begin{equation}
\label{eq::PWL_2D_centroid}
	\vec{r}_{c} =  \sum\displaylimits_{j=1}^{n} \alpha_{j} \vec{r}_j ,
\end{equation}

\noindent where $\alpha_{j}$ are the vertex weights functions and, 

\begin{equation}
\label{eq::PWL_2D_vertex_weight_sum}
 \sum\displaylimits_{j=1}^{n} \alpha_{j} = 1.
\end{equation}

\noindent For this work, we continue to use the definition for the vertex weight functions from previous works \cite{ref::PWLD_stone_adams,ref::PWLD_stone_adams_unstructured,bailey2008phd}: $\alpha_{j}= 1/n$. Therefore, the vertex weight functions are the same for every vertex, and the cell centroid simply becomes the average position of all the vertices. Using these vertex weight functions, the PWL basis function for vertex $j$, $\lambda_j^{PWL}$, is defined as

\begin{equation}
\label{eq::PWL_2D}
\lambda_j^{PWL} (\vec{r}) = t_j (\vec{r}) + \alpha_j t_c (\vec{r}) .
\end{equation}

\noindent In Eq. (\ref{eq::PWL_2D}), $t_j$ is the standard 2D linear function with unity at vertex $j$ that linearly decreases to zero at the cell center and each adjoining vertex. $t_c$ is the 2D cell ``tent'' function located at $\vec{r}_{c}$ which is unity at the cell center and linearly decreases to zero at each cell vertex. $\alpha_{j}$ is the weight parameter for vertex $j$. The functional form of Eq. (\ref{eq::PWL_2D}) with identical vertex weights means that the PWL function for vertex $j$, within the domain of the polygon, linearly decreases to a value of $1/n$ at the polygonal center. From there, the function linearly decreases to zero on all faces that are not connected to vertex $j$.

%------------------------------------------------------------------------------------------------------------
\subsection{Mean Value Coordinates}
%------------------------------------------------------------------------------------------------------------

The third linearly-complete 2D polygonal funcitons that we will analyze are the mean value coordinates (MV) \cite{floater2003mean,floater2015generalized}. These functions were derived to approximate harmonic maps on a polygon by a set of piecewise linear maps over a triangulation of the polygon for use in computer aided graphic design. The mean value function at vertex $j$, $\lambda_{j}^{MV}$, is defined as

\begin{equation}
\label{eq::MV_BF}
\lambda_{j}^{MV} (\vec{r}) = \frac{w_j (\vec{r}) }{\sum\limits_{i=1}^{n} w_i (\vec{r})} ,
\end{equation}

\noindent where the mean value weight function for vertex $j$, $w_j$, has the following definition:

\begin{equation}
\label{eq::MV_weights}
w_j (\vec{r})  = \frac{\tan(\alpha_{j-1} / 2) + \tan(\alpha_j / 2)}{|\vec{r}_j - \vec{r}|}.
\end{equation}

%------------------------------------------------------------------------------------------------------------
\subsection{Maximum Entropy Coordinates}
%------------------------------------------------------------------------------------------------------------

\tcr{short history and functional form}

\begin{equation}
\label{eq::ME_BF}
\lambda_{j}^{ME} (\vec{r}) = \frac{w_j (\vec{r}) }{\sum\limits_{i=1}^{n} w_i (\vec{r})} ,
\end{equation}

\noindent where the maximum entropy weight function for vertex $j$, $w_j$, has the following definition,

\begin{equation}
\label{eq::ME_weights}
w_j (\vec{r})  = m_j(\vec{r}) \exp(-  \vec{\kappa} \cdot (\vec{r}_j - \vec{r})),
\end{equation}

\noindent and $\vec{\kappa}$ is a vector value of dimension 2 that will be explained shortly. In the context of Eq. (\ref{eq::ME_weights}), the prior distribution, $m_j$, can be viewed as a weight function associated with vertex $j$. This means that there is variability that one can employ for these weight functions. For FEM applications, an appropriate functional form for the prior distribution is given by

\begin{equation}
\label{eq::ME_prior_funcs}
 m_j(\vec{r}) = \frac{\pi_j (\vec{r}) }{\sum\limits_{k=1}^{n} \pi_k (\vec{r})},
\end{equation}

\noindent where

\begin{equation}
\label{eq::ME_prior_products}
\pi_j (\vec{r}) = \prod\limits_{k \neq j-1, j}^{n} \rho_k (\vec{r})  = \rho_1(\vec{r}) \ldots \rho_{j-2}(\vec{r}) \rho_{j+1}(\vec{r}) \ldots \rho_{n}(\vec{r}) ,
\end{equation}

\noindent and

\begin{equation}
\label{eq::ME_face_funcs}
\rho_j (\vec{x}) = || \vec{x} - \vec{x}_j || + || \vec{x} - \vec{x}_{j+1} || - || \vec{x}_{j+1} - \vec{x}_j || .
\end{equation}

%------------------------------------------------------------------------------------------------------------
\subsection{Summary of the Linear Polygonal Basis Functions}
%------------------------------------------------------------------------------------------------------------

\tcr
{
\begin{enumerate}
\item table with summary
\item example contour plots (small)
\end{enumerate}
}

\begin{table}[hbt]
\centering
\caption{Summary of the properties of the 2D coordinate systems used on polygons.}
\begin{tabular}{|c|c|c|c|c|}
\hline
Basis Function & Dimension & Polygon Type & Integration & Evaluation \\
\hline \hline
Wachspress	&2D/3D&	Convex&	Numerical	&Direct\\ \hline
PWL&	1D/2D/3D&	Convex/Concave&	Analytical	&Direct\\ \hline
Mean Value&	2D&	Convex/Concave&	Numerical	&Direct\\ \hline
Max Entropy&	1D/2D/3D	&Convex/Concave&	Numerical&	Iterative\\ \hline
\end{tabular}
\label{tab::2Dlin_summary}
\end{table}

% Linear contour plots
% -----------------------
\begin{figure}[hbt]
\centering
{
	\begin{subfigure}[b]{0.325\textwidth}
		\centering
		\includegraphics[width=\textwidth]{\BFfigpath{square_WACHSPRESS1_contour_b4.png}}
		\caption{Wachspress}
	\end{subfigure}
	\hspace{1.25cm}
	\begin{subfigure}[b]{0.325\textwidth}
		\centering
		\includegraphics[width=\textwidth]{\BFfigpath{square_PWLD1_contour_b4.png}}
		\caption{PWL}
	\end{subfigure}
}
	\vspace{3mm}
{
	\begin{subfigure}[b]{0.325\textwidth}
		\centering
		\includegraphics[width=\textwidth]{\BFfigpath{square_MV1_contour_b4.png}}
		\caption{Mean Value}
	\end{subfigure}
	\hspace{1.25cm}
	\begin{subfigure}[b]{0.325\textwidth}
		\centering
		\includegraphics[width=\textwidth]{\BFfigpath{square_MAXENT1_contour_b4.png}}
		\caption{Maximum Entropy}
	\end{subfigure}
}
\caption{Contour plots of the different linear basis function on the unit square located at vertex (0,1).}
\label{fig::Linear_unit_square_basis_functions}
\end{figure}

%%%%%%%%%%%%%%%%%%%%%%%%%%%%%%%%%%%%%%%%%%%%%%%%%%%%%%%%%%%%%%%%%%%%
%%%%%%%%%%%%%%%%%%%%%%%%%%%%%%%%%%%%%%%%%%%%%%%%%%%%%%%%%%%%%%%%%%%%
\section{Quadratic Serendipity Polygonal Basis Functions} \label{sec::quadpoly}
%%%%%%%%%%%%%%%%%%%%%%%%%%%%%%%%%%%%%%%%%%%%%%%%%%%%%%%%%%%%%%%%%%%%
%%%%%%%%%%%%%%%%%%%%%%%%%%%%%%%%%%%%%%%%%%%%%%%%%%%%%%%%%%%%%%%%%%%%

\tcr
{
\begin{enumerate}
\item Image of conversion procedure
\item conversion methodology
\item example contour plots
\end{enumerate}
}

% serendipity overview figure
% ---------
\begin{figure}[hbt]
\centering
\includegraphics[width=0.90\textwidth]{\BFfigpath{linear_to_quad_process.png}}
\caption{Overview of the process to construct the quadratic serendipity basis functions on polygons. The filled dots correspond to basis functions that maintain the Lagrange property while empty dots do not.}
\label{fig::BF_2D_quad_process}
\end{figure}
% ---------

% Quadratic contour plots - b4
% ----------------------------------
\begin{figure}[hbt]
\centering
{
	\begin{subfigure}[b]{0.30\textwidth}
		\centering
		\includegraphics[width=0.9\textwidth]{\BFfigpath{square_WACHSPRESS2_contour_b4.png}}
		\caption{Wachspress}
	\end{subfigure}
	\hspace{1.25cm}
	\begin{subfigure}[b]{0.30\textwidth}
		\centering
		\includegraphics[width=0.9\textwidth]{\BFfigpath{square_PWLD2_contour_b4.png}}
		\caption{PWL}
	\end{subfigure}
}
	\vspace{3mm}
{
	\begin{subfigure}[b]{0.30\textwidth}
		\centering
		\includegraphics[width=0.9\textwidth]{\BFfigpath{square_MV2_contour_b4.png}}
		\caption{Mean Value}
	\end{subfigure}
	\hspace{1.25cm}
	\begin{subfigure}[b]{0.30\textwidth}
		\centering
		\includegraphics[width=0.9\textwidth]{\BFfigpath{square_MAXENT2_contour_b4.png}}
		\caption{Maximum Entropy}
	\end{subfigure}
}
\caption{Contour plots of the different quadratic serendipity basis function on the unit square located at vertex (0,1).}
\label{fig::Quad_unit_square_basis_functions_b4}
\end{figure}

% Quadratic contour plots - b8
% ----------------------------------
\begin{figure}[hbt]
\centering
{
	\begin{subfigure}[b]{0.30\textwidth}
		\centering
		\includegraphics[width=0.9\textwidth]{\BFfigpath{square_WACHSPRESS2_contour_b8.png}}
		\caption{Wachspress}
	\end{subfigure}
	\hspace{1.25cm}
	\begin{subfigure}[b]{0.30\textwidth}
		\centering
		\includegraphics[width=0.9\textwidth]{\BFfigpath{square_PWLD2_contour_b8.png}}
		\caption{PWL}
	\end{subfigure}
}
	\vspace{3mm}
{
	\begin{subfigure}[b]{0.30\textwidth}
		\centering
		\includegraphics[width=0.9\textwidth]{\BFfigpath{square_MV2_contour_b8.png}}
		\caption{Mean Value}
	\end{subfigure}
	\hspace{1.25cm}
	\begin{subfigure}[b]{0.30\textwidth}
		\centering
		\includegraphics[width=0.9\textwidth]{\BFfigpath{square_MAXENT2_contour_b8.png}}
		\caption{Maximum Entropy}
	\end{subfigure}
}
\caption{Contour plots of the different quadratic serendipity basis function on the unit square located at vertex (0,1/2).}
\label{fig::Quad_unit_square_basis_functions_b8}
\end{figure}



%%%%%%%%%%%%%%%%%%%%%%%%%%%%%%%%%%%%%%%%%%%%%%%%%%%%%%%%%%%%%%%%%%%%
%%%%%%%%%%%%%%%%%%%%%%%%%%%%%%%%%%%%%%%%%%%%%%%%%%%%%%%%%%%%%%%%%%%%
\section{Adaptive Mesh Refinement Using Polygonal Basis Functions} \label{sec::amr}
%%%%%%%%%%%%%%%%%%%%%%%%%%%%%%%%%%%%%%%%%%%%%%%%%%%%%%%%%%%%%%%%%%%%
%%%%%%%%%%%%%%%%%%%%%%%%%%%%%%%%%%%%%%%%%%%%%%%%%%%%%%%%%%%%%%%%%%%%

\tcr{overview of AMR for transport calculations}

\iffalse
%%%%%%%%%%%%%%%%%%%%%%%%%%%%%%%%%%%%%%%%%%%%%%%%%%%%%%%%%%%%%%%%%%%%
%%%%%%%%%%%%%%%%%%%%%%%%%%%%%%%%%%%%%%%%%%%%%%%%%%%%%%%%%%%%%%%%%%%%
\section{Numerical Results} \label{sec::results}
%%%%%%%%%%%%%%%%%%%%%%%%%%%%%%%%%%%%%%%%%%%%%%%%%%%%%%%%%%%%%%%%%%%%
%%%%%%%%%%%%%%%%%%%%%%%%%%%%%%%%%%%%%%%%%%%%%%%%%%%%%%%%%%%%%%%%%%%%

%------------------------------------------------------------------------------------------------------------
\subsection{Exactly-Linear Transport Solutions}
%------------------------------------------------------------------------------------------------------------

Our first numerical verification example demonstrates that the linear polygonal finite element basis functions capture an exactly-linear solution space. We will show this by the method of exact solutions (MES). We build our exact solution by investigating the 2D, 1 energy group transport problem with no scattering and an angle-dependent distributed source,

\begin{equation}
\label{eq::Results_Linear_angflux}
\mu \frac{\partial \psi}{\partial x} + \eta \frac{\partial \psi}{\partial y} + \sigma_t \psi = Q(x,y, \mu, \eta), 
\end{equation}

\noindent where the streaming term was separated into the corresponding two-dimensional terms. We then define an angular flux solution that is linear in both space and angle along with the corresponding 0th moment scalar flux ($\phi_{0,0} \rightarrow \phi$) solution:

\begin{equation}
\label{eq::BF_Results_Linear_fluxsols}
\begin{aligned}
\psi (x,y,\mu,\eta) &= ax + by + c \mu + d\eta + e\\
\phi (x,y) &= 2 \pi \left( ax + by  + e \right)
\end{aligned} .
\end{equation}

% Linear contour plots
\begin{figure}
\centering
{
	\begin{subfigure}[b]{0.45\textwidth}
		\centering
		\label{subfig::cart_wach_lin_sol}
		\includegraphics[width=\textwidth]{\BFfigpath{cart_WACHSPRESS_k1.eps}}
		\caption{Cartesian}
	\end{subfigure}
	\hfill
	\begin{subfigure}[b]{0.45\textwidth}
		\centering
		\label{subfig::tri_wach_lin_sol}
		\includegraphics[width=\textwidth]{\BFfigpath{tri_WACHSPRESS_k1.eps}}
		\caption{Triangular}
	\end{subfigure}
}
\vspace{3mm}
{
	\begin{subfigure}[b]{0.45\textwidth}
		\centering
		\label{subfig::shes_quad_wach_lin_sol}
		\includegraphics[width=\textwidth]{\BFfigpath{shes_poly_WACHSPRESS_k1.eps}}
		\caption{Shestakov Polygons}
	\end{subfigure}
	\hfill
	\begin{subfigure}[b]{0.45\textwidth}
		\centering
		\label{subfig::smooth_poly_wach_lin_sol}
		\includegraphics[width=\textwidth]{\BFfigpath{smooth_poly_WACHSPRESS_k1.eps}}
		\caption{Shestakov Polygons}
	\end{subfigure}
}
\vspace{3mm}
{
	\begin{subfigure}[b]{0.45\textwidth}
		\centering
		\label{subfig::z_quad_wach_lin_sol}
		\includegraphics[width=\textwidth]{\BFfigpath{z_quad_WACHSPRESS_k1.eps}}
		\caption{Z-Quadrilaterals}
	\end{subfigure}
	\hfill
	\begin{subfigure}[b]{0.45\textwidth}
		\centering
		\label{subfig::z_poly_wach_lin_sol}
		\includegraphics[width=\textwidth]{\BFfigpath{z_poly_WACHSPRESS_k1.eps}}
		\caption{Z-Polygons}
	\end{subfigure}
}
\caption{Contour plots of the exactly-linear solution with the Wachspress basis functions.}
\label{fig::Results_Linear_wach_sol}
\end{figure}

%------------------------------------------------------------------------------------------------------------
\subsection{Exactly-Quadratic Transport Solutions}
%------------------------------------------------------------------------------------------------------------

The first problem interpolates the $\{ 1, x, y, x^2, xy, y^2 \}$ span of functions, which we denote with the following general and exactly-quadratic solution, $\{\psi_q, \phi_q\}$:

\begin{equation}
\label{eq::Results_Quadratic_fluxsols}
\begin{aligned}
\psi_q (x,y,\mu,\eta) &= a + bx + c y+ d xy + e x^2 + fy^2 \\
\phi_q (x,y) &= 2 \pi \left(  a + bx + c y+ d xy + e x^2 + fy^2 \right)
\end{aligned} 
\end{equation}

\noindent We clearly see that any combination of positive or negative (non-zero) values for the coefficients $a-f$ will span the quadratic serendipity space. The second problem contains terms up to $x^2 y^2$. This higher-order functional form has a solution, $\{ \psi_{x2y2}, \phi_{x2y2}\}$, given by the following

\begin{equation}
\label{eq::Results_x2y2_fluxsols}
\begin{aligned}
\psi_{x2y2} (x,y,\mu,\eta) &= x \left(L_x - x \right) y \left(L_y - y \right) \\
\phi_{x2y2} (x,y) &= 2 \pi x \left(L_x - x \right) y \left(L_y - y \right) 
\end{aligned} ,
\end{equation}

\noindent where $L_x$ and $L_y$ are the dimensions of the domain in $x$ and $y$.

% 1,x,y,x^2,xy,y^2 table
\begin{table}[hbt]
\caption{$L_2$-norm of the error in the quadratic solution spanning $\{ 1, x, y, x^2, xy, y^2 \}$ for the different quadratic serendipity basis functions on different mesh types.}
\centering
\def\arraystretch{1.25}
\begin{tabular}{|c|c|c|c|c|}
\hline
& \multicolumn{4}{c}{Basis Functions}\vline\\
\hline
Mesh Type & Wachspress & PWL& Mean Value& Max. Entropy \\
\hline
Cartesian&2.23e-13&7.25e-13&5.68e-14&5.65e-14\\
Triangular&7.85e-14&1.46e-13&2.52e-14&2.54e-13\\
Shes. Poly&1.14e-14&6.82e-14&5.75e-14&1.25e-13\\
Sine Poly&2.56e-13&4.15e-13&3.25e-13&6.37e-13\\
Z-Poly&5.24e-14&5.35e-14&1.68e-13&5.19e-14\\
Z-Quad&6.64e-14&4.42e-14&8.29e-14&6.82e-13\\
\hline
\end{tabular}
\end{table}

%
\begin{table}[hbt]
\caption{$L_2$-norm of the error in the quadratic solution containing the $x^2y^2$ term for the different quadratic serendipity basis functions on different mesh types.}
\centering
\def\arraystretch{1.25}
\begin{tabular}{|c|c|c|c|c|}
\hline
& \multicolumn{4}{c}{Basis Functions}\vline\\
\hline
Mesh Type & Wachspress & PWL& Mean Value& Max. Entropy \\
\hline
Cartesian&3.50e-06&2.72e-05&8.29e-06&3.86e-05\\
Triangular&5.13e-05&5.13e-05&5.13e-05&5.13e-05\\
Shes. Poly&3.97e-04&3.37e-04&2.81e-04&3.91e-04\\
Sine Poly&3.75e-05&1.22e-04&7.62e-05&1.39e-04\\
Z-Poly&2.93e-05&3.07e-05&2.59e-05&3.46e-05\\
Z-Quad&2.98e-05&8.73e-05&5.08e-05&1.17e-04\\
\hline
\end{tabular}
\end{table}

%------------------------------------------------------------------------------------------------------------
\subsection{Convergence Rate Analysis by the Method of Manufactured Solutions}
%------------------------------------------------------------------------------------------------------------
\tcr{no mention of irr}

The sinusoid flux solutions, \{$\psi_s$, $\phi_s$\}, have the following parameterized form,

\begin{equation}
\label{eq::Results_MMS_sinefluxsols}
\begin{aligned}
\psi_s (x,y) = &\sin(\nu  \frac{\pi x}{L_x}) \sin(\nu  \frac{\pi y}{L_y}), \\ 
\phi_s (x,y) = 2 \pi &\sin(\nu  \frac{\pi x}{L_x}) \sin(\nu  \frac{\pi y}{L_y}),
\end{aligned} 
\end{equation}

\noindent where $\nu$ is a frequency parameter. We restrict this parameter to positive integers ($\nu = 1,2,3,...$). The Gaussian solution space, \{$\psi_g$, $\phi_g$\}, that has its local maximum centered at ($x_0,y_0$) has the parameterized form,

\begin{equation}
\label{eq::BF_Results_MMS_gaussfluxsols}
\begin{aligned}
\psi_g (x,y) = &C_M x (L_x - x) y (L_y - y) \exp(-\frac{(x-x_0)^2 + (y-y_0)^2}{\gamma}), \\ 
\phi_g (x,y) = 2 \pi &C_M x (L_x - x) y (L_y - y) \exp(-\frac{(x-x_0)^2 + (y-y_0)^2}{\gamma}),
\end{aligned} 
\end{equation}

\noindent where the constants in the equations are:

\begin{equation}
\label{eq::BF_Results_MMS_gaussconsts}
C_M = \frac{100}{L_x^2 L_y^2} \qquad \gamma = \frac{L_x L_y}{100} .
\end{equation}

%------------------------------------------------------------------------------------------------------------
\subsection{Convergence Rate Analysis Bounded by the Solution Regularity}
%------------------------------------------------------------------------------------------------------------
\tcr{summarize results; give poly only for cv rate=1/2. put the 3/2 in appendix.}



% Left-face incidence mesh and solution
% ---------------------------------------------
\begin{figure}
\centering
	\begin{subfigure}[b]{0.425\textwidth}
		\centering
		\label{subfig::PA_MeshSol_Poly}
		\includegraphics[width=\textwidth]{\BFfigpath{PALeftSol_SplitPoly.png}}
		\caption{Split-Polygonal Mesh}
	\end{subfigure}
	\hfill
	\begin{subfigure}[b]{0.425\textwidth}
		\centering
		\label{subfig::PA_MeshSol_SplitPoly}
		\includegraphics[width=\textwidth]{\BFfigpath{PALeftSol_Poly.png}}
		\caption{Polygonal Mesh}
	\end{subfigure}
\caption{Example solution of the purely-absorbing medium case with left-face incidence and $\sigma_t=1$ using the linear PWL basis functions.}
\label{fig::Results_PA_Left_Solutions}
\end{figure}

% p+1split-polygonal convergence rate plots - left-face incidence
% ---------------------------------------------------------------------------
\begin{figure}
\centering
{
	\begin{subfigure}[b]{0.485\textwidth}
		\centering
		\label{subfig::PA_Left_SplitPoly_sig1}
		\includegraphics[width=\textwidth]{\BFfigpath{PAErr_Left_SplitPoly_sig1.eps}}
	\caption{$\sigma_t = 1$}
	\end{subfigure}
	\hfill
	\begin{subfigure}[b]{0.485\textwidth}
		\centering
		\label{subfig::PA_Left_SplitPoly_sig10}
		\includegraphics[width=\textwidth]{\BFfigpath{PAErr_Left_SplitPoly_sig10.eps}}
	\caption{$\sigma_t = 10$}
	\end{subfigure}
}
\vspace{1cm}
{
	\begin{subfigure}[b]{0.485\textwidth}
		\centering
		\label{subfig::PA_Left_SplitPoly_sig50}
		\includegraphics[width=\textwidth]{\BFfigpath{PAErr_Left_SplitPoly_sig50.eps}}
	\caption{$\sigma_t = 50$}
	\end{subfigure}
	\hfill
	\begin{subfigure}[b]{0.485\textwidth}
		\centering
		\label{subfig::PA_Left_SplitPoly_sig100}
		\includegraphics[width=\textwidth]{\BFfigpath{PAErr_Left_SplitPoly_sig100.eps}}
	\caption{$\sigma_t = 100$}
	\end{subfigure}
}
\caption{Convergence rates for the pure absorber problem with left-face incidence on split-polygonal meshes with different values of $\sigma_t$.}
\label{fig::Results_PA_Left_SplitPoly}
\end{figure}


% 1/2 polygonal convergence rate plots - left-face incidence
% ---------------------------------------------------------------------
\begin{figure}
\centering
{
	\begin{subfigure}[b]{0.485\textwidth}
		\centering
		\label{subfig::PA_Left_Poly_sig1}
		\includegraphics[width=\textwidth]{\BFfigpath{PAErr_Left_Poly_sig1.eps}}
	\caption{$\sigma_t = 1$}
	\end{subfigure}
	\hfill
	\begin{subfigure}[b]{0.485\textwidth}
		\centering
		\label{subfig::PA_Left_Poly_sig10}
		\includegraphics[width=\textwidth]{\BFfigpath{PAErr_Left_Poly_sig10.eps}}
	\caption{$\sigma_t = 10$}
	\end{subfigure}
}
\vspace{1cm}
{
	\begin{subfigure}[b]{0.485\textwidth}
		\centering
		\label{subfig::PA_Left_Poly_sig50}
		\includegraphics[width=\textwidth]{\BFfigpath{PAErr_Left_Poly_sig50.eps}}
	\caption{$\sigma_t = 50$}
	\end{subfigure}
	\hfill
	\begin{subfigure}[b]{0.485\textwidth}
		\centering
		\label{subfig::PA_Left_Poly_sig100}
		\includegraphics[width=\textwidth]{\BFfigpath{PAErr_Left_Poly_sig100.eps}}
	\caption{$\sigma_t = 100$}
	\end{subfigure}
}
\caption{Convergence rates for the pure absorber problem with left-face incidence on polygonal meshes with different values of $\sigma_t$.}
\label{fig::Results_PA_Left_Poly}
\end{figure}

%------------------------------------------------------------------------------------------------------------
\subsection{Thick Diffusion Limit}
%------------------------------------------------------------------------------------------------------------
\tcr{\begin{itemize}
\item 1D unresolved boundary layer
\item 2D from dissertation
\end{itemize}}
\fi

%%%%%%%%%%%%%%%%%%%%%%%%%%%%%%%%%%%%%%%%%%%%%%%%%%%%%%%%%%%%%%%%%%%%
%%%%%%%%%%%%%%%%%%%%%%%%%%%%%%%%%%%%%%%%%%%%%%%%%%%%%%%%%%%%%%%%%%%%
\section{Conclusions} \label{sec::conclusions}
%%%%%%%%%%%%%%%%%%%%%%%%%%%%%%%%%%%%%%%%%%%%%%%%%%%%%%%%%%%%%%%%%%%%
%%%%%%%%%%%%%%%%%%%%%%%%%%%%%%%%%%%%%%%%%%%%%%%%%%%%%%%%%%%%%%%%%%%%
In this paper...

%%%%%%%%%%%%%%%%%%%%%%%%%%%%%%%%%%%%%%%%%%%%%%%%%%%%%%%%%%%%%%%%%%%%
%%%%%%%%%%%%%%%%%%%%%%%%%%%%%%%%%%%%%%%%%%%%%%%%%%%%%%%%%%%%%%%%%%%%
\section*{Acknowledgments} 
%%%%%%%%%%%%%%%%%%%%%%%%%%%%%%%%%%%%%%%%%%%%%%%%%%%%%%%%%%%%%%%%%%%%
%%%%%%%%%%%%%%%%%%%%%%%%%%%%%%%%%%%%%%%%%%%%%%%%%%%%%%%%%%%%%%%%%%%%
This research was performed under appointment to the Rickover Graduate Fellowship Program in Nuclear Engineering sponsored by the Naval Reactors Division of the United States Department of Energy.

%%%%%%%%%%%%%%%%%%%%%%%%%%%%%%%%%%%%%%%%%%%%%%%%%%%%%%%%%%%%%%%%%%%%
%%%%%%%%%%%%%%%%%%%%%%%%%%%%%%%%%%%%%%%%%%%%%%%%%%%%%%%%%%%%%%%%%%%%
%%%%%%%%%%%%%%%%%%%%%%%%%%%%%%%%%%%%%%%%%%%%%%%%%%%%%%%%%%%%%%%%%%%%
%%%%%%%%%%%%%%%%%%%%%%%%%%%%%%%%%%%%%%%%%%%%%%%%%%%%%%%%%%%%%%%%%%%%

\bibliographystyle{unsrt}
\bibliography{references}

% Begin appendices
\appendix
%%%%%%%%%%%%%%%%%%%%%%%%%%%%%%%%%%%%%%%%%%%%%%%%%%%%%%%%%%%%%%%%%%%%
%%%%%%%%%%%%%%%%%%%%%%%%%%%%%%%%%%%%%%%%%%%%%%%%%%%%%%%%%%%%%%%%%%%%
\section{Limits of the Barycentric Functions on the Boundary}  \label{app::bound}
%%%%%%%%%%%%%%%%%%%%%%%%%%%%%%%%%%%%%%%%%%%%%%%%%%%%%%%%%%%%%%%%%%%%
%%%%%%%%%%%%%%%%%%%%%%%%%%%%%%%%%%%%%%%%%%%%%%%%%%%%%%%%%%%%%%%%%%%%

%%%%%%%%%%%%%%%%%%%%%%%%%%%%%%%%%%%%%%%%%%%%%%%%%%%%%%%%%%%%%%%%%%%%
%%%%%%%%%%%%%%%%%%%%%%%%%%%%%%%%%%%%%%%%%%%%%%%%%%%%%%%%%%%%%%%%%%%%
\section{Conversion of the Serendipity Functions to the Lagrange Space}  \label{app::Lagrange}
%%%%%%%%%%%%%%%%%%%%%%%%%%%%%%%%%%%%%%%%%%%%%%%%%%%%%%%%%%%%%%%%%%%%
%%%%%%%%%%%%%%%%%%%%%%%%%%%%%%%%%%%%%%%%%%%%%%%%%%%%%%%%%%%%%%%%%%%%

\iffalse
%%%%%%%%%%%%%%%%%%%%%%%%%%%%%%%%%%%%%%%%%%%%%%%%%%%%%%%%%%%%%%%%%%%%
%%%%%%%%%%%%%%%%%%%%%%%%%%%%%%%%%%%%%%%%%%%%%%%%%%%%%%%%%%%%%%%%%%%%
\section{Additional Convergence Rate Analysis for Continuous Transport Solutions Bounded by the Solution Regularity}  \label{app::Lagrange}
%%%%%%%%%%%%%%%%%%%%%%%%%%%%%%%%%%%%%%%%%%%%%%%%%%%%%%%%%%%%%%%%%%%%
%%%%%%%%%%%%%%%%%%%%%%%%%%%%%%%%%%%%%%%%%%%%%%%%%%%%%%%%%%%%%%%%%%%%


% Left-face and top-face incidence mesh and solution
% ---------------------------------------------
\begin{figure}
\centering
	\begin{subfigure}[b]{0.425\textwidth}
		\centering
		\label{subfig::PA_MeshSol_Poly}
		\includegraphics[width=\textwidth]{\BFfigpath{PALeftTopSol_SplitPoly.png}}
		\caption{Split-Polygonal Mesh}
	\end{subfigure}
	\hfill
	\begin{subfigure}[b]{0.425\textwidth}
		\centering
		\label{subfig::PA_MeshSol_SplitPoly}
		\includegraphics[width=\textwidth]{\BFfigpath{PALeftTopSol_Poly.png}}
		\caption{Polygonal Mesh}
	\end{subfigure}
\caption{Example solution of the purely-absorbing medium case with left-face and top-face incidence and $\sigma_t=1$ using the linear PWL basis functions.}
\label{fig::Results_PA_Left_Solutions}
\end{figure}

% p+1split-polygonal convergence rate plots - left-face incidence
% ---------------------------------------------------------------------------
\begin{figure}
\centering
{
	\begin{subfigure}[b]{0.485\textwidth}
		\centering
		\label{subfig::PA_LeftTop_SplitPoly_sig1}
		\includegraphics[width=\textwidth]{\BFfigpath{PAErr_LeftTop_SplitPoly_sig1.eps}}
	\caption{$\sigma_t = 1$}
	\end{subfigure}
	\hfill
	\begin{subfigure}[b]{0.485\textwidth}
		\centering
		\label{subfig::PA_LeftTop_SplitPoly_sig10}
		\includegraphics[width=\textwidth]{\BFfigpath{PAErr_LeftTop_SplitPoly_sig10.eps}}
	\caption{$\sigma_t = 10$}
	\end{subfigure}
}
\vspace{1cm}
{
	\begin{subfigure}[b]{0.485\textwidth}
		\centering
		\label{subfig::PA_LeftTop_SplitPoly_sig50}
		\includegraphics[width=\textwidth]{\BFfigpath{PAErr_LeftTop_SplitPoly_sig50.eps}}
	\caption{$\sigma_t = 50$}
	\end{subfigure}
	\hfill
	\begin{subfigure}[b]{0.485\textwidth}
		\centering
		\label{subfig::PA_LeftTop_SplitPoly_sig100}
		\includegraphics[width=\textwidth]{\BFfigpath{PAErr_LeftTop_SplitPoly_sig100.eps}}
	\caption{$\sigma_t = 100$}
	\end{subfigure}
}
\caption{Convergence rates for the pure absorber problem with left-face and top-face incidence on split-polygonal meshes with different values of $\sigma_t$.}
\label{fig::Results_PA_Left_SplitPoly}
\end{figure}


% 1/2 polygonal convergence rate plots - left-face incidence
% ---------------------------------------------------------------------
\begin{figure}
\centering
{
	\begin{subfigure}[b]{0.485\textwidth}
		\centering
		\label{subfig::PA_LeftTop_Poly_sig1}
		\includegraphics[width=\textwidth]{\BFfigpath{PAErr_LeftTop_Poly_sig1.eps}}
	\caption{$\sigma_t = 1$}
	\end{subfigure}
	\hfill
	\begin{subfigure}[b]{0.485\textwidth}
		\centering
		\label{subfig::PA_LeftTop_Poly_sig10}
		\includegraphics[width=\textwidth]{\BFfigpath{PAErr_LeftTop_Poly_sig10.eps}}
	\caption{$\sigma_t = 10$}
	\end{subfigure}
}
\vspace{1cm}
{
	\begin{subfigure}[b]{0.485\textwidth}
		\centering
		\label{subfig::PA_LeftTop_Poly_sig50}
		\includegraphics[width=\textwidth]{\BFfigpath{PAErr_LeftTop_Poly_sig50.eps}}
	\caption{$\sigma_t = 50$}
	\end{subfigure}
	\hfill
	\begin{subfigure}[b]{0.485\textwidth}
		\centering
		\label{subfig::PA_LeftTop_Poly_sig100}
		\includegraphics[width=\textwidth]{\BFfigpath{PAErr_LeftTop_Poly_sig100.eps}}
	\caption{$\sigma_t = 100$}
	\end{subfigure}
}
\caption{Convergence rates for the pure absorber problem with left-face and top-face incidence on polygonal meshes with different values of $\sigma_t$.}
\label{fig::Results_PA_Left_Poly}
\end{figure}

\fi
%%%%%%%%%%%%%%%%%%%%%%%%%%%%%%%%%%%%%%%%%%%%%%%%%%%%%%%%%%%%%%%%%%%%
%%%%%%%%%%%%%%%%%%%%%%%%%%%%%%%%%%%%%%%%%%%%%%%%%%%%%%%%%%%%%%%%%%%%
\end{document}
%%%%%%%%%%%%%%%%%%%%%%%%%%%%%%%%%%%%%%%%%%%%%%%%%%%%%%%%%%%%%%%%%%%%
%%%%%%%%%%%%%%%%%%%%%%%%%%%%%%%%%%%%%%%%%%%%%%%%%%%%%%%%%%%%%%%%%%%%
