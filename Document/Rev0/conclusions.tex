%%%%%%%%%%%%%%%%%%%%%%%%%%%%%%%%%%%%%%%%%%%%%%%%%%%
%
%  New template code for TAMU Theses and Dissertations starting Fall 2012.  
%  For more info about this template or the 
%  TAMU LaTeX User's Group, see http://www.howdy.me/.
%
%  Author: Wendy Lynn Turner 
%	 Version 1.0 
%  Last updated 8/5/2012
%
%%%%%%%%%%%%%%%%%%%%%%%%%%%%%%%%%%%%%%%%%%%%%%%%%%%
%%%                           Conclusions
%%%%%%%%%%%%%%%%%%%%%%%%%%%%%%%%%%%%%%%%%%%%%%%%%%%
\chapter{\uppercase {Conclusions}}
\label{sec::Conclusions}


%%%%%%%%%%%%%%%%%%%%%%%%%%%%%%%%%%%%%%%%%%%%%%%%%%%
%%%   Section - Conclusions
\section{Conclusions}
\label{sec::Conclusions_Conclusions}

In this dissertation, we have performed work to advance the state-of-the-art in solving the DGFEM $S_N$ transport equation on polytope meshes using massively-parallel computer architectures. We have done this by investigating two different topical areas. First, we investigated four different linearly-complete polygonal coordinate systems to be used as FEM basis functions. Then, we investigated the methodology to convert these linearly-complete functions into the quadratic serendipity space of functions. Higher-order FEM basis functions have two-fold benefits. One, they achieve enhanced convergence rates with mesh refinement due to their higher-dimensional interpolation space. Two, they can enhance parallel computing efficiancy by decreasing memory-access bottlenecks by giving the CPUs more data to compute per solve. Currently, memory-access calls are the limiting bottleneck on massively-parallel computer architectures. Following the work on the quadratic serendipity basis functions on polygonal meshes, we analyzed the performance of 1-group and multigroup thermal neutron upscattering DSA acceleration methods. Specifically, we sought to analyze 

We tested both of these topical areas through theoretical and numerical analyses. Our conclusions involving the work on the higher-order polygonal basis functions are listed below:

\begin{enumerate}
\item 
\item 
\item 
\end{enumerate}

Next, our conclusions for the DSA work on massively-parallel computer architectures are listed below:

\begin{enumerate}
\item The Symmetric Interior Penalty (SIP) discretization of the diffusion equation is an efficient solver for the 3D diffusion problem using DFEM. It is Symmetric Positive Definite (SPD) and its system matrix is easily solvable with a Preconditioned Conjugate Gradient Method (PCG). We demonstrated that linearly-complete polyhedral basis functions can capture exactly-linear diffusion solutions on arbitrary grids. We also showed that the proper second-order convergence is obtained in the $L_2$-norm.
\item The Modified Interior Penalty (MIP) method is a modification to the SIP scheme that can be used for as a DFEM diffusion form for DSA calcualtions. It is also SPD and easily solvable the same as SIP. Fourier and numerical analysis showed that it is efficient and robust on parallelipipeds, even those with high aspect ratios. However, this efficiency degrades with large discontinuities across material cross sections.
\item MIP DSA was successfully implemented in the Texas A\&M University massively-parallel $S_N$ sweeping code: PDT.
\item We tested three different 
\end{enumerate}

%%%%%%%%%%%%%%%%%%%%%%%%%%%%%%%%%%%%%%%%%%%%%%%%%%%
%%%   Section - Conclusions
\section{Open Items}
\label{sec::Conclusions_Open_Items}

While the work in this dissertation answered several open questions related to the calculation of the DGFEM $S_N$ transport equation on massively-parallel architectures, several items remain for ongoing research. We now list the open items that we have identified:

\begin{enumerate}
\item {\em Quadratic serendipity basis functions on 3D polyhedra:} \\
The direct extension of the work involving the 2D quadratic serendipity basis functions would be to form quadratically-complete, analogous serendipity coordinates for arbitrary 3D polyhedra. To maintain quadratic completeness in 3D, the coordinates would be beholden to the ten quadratic 3D constraints which would require exact interpolation of the $\{ 1,x,y,z,xy,xz,yz,x^2,y^2,z^2 \}$ span of functions. Along a polyhedral face, the methodology presented in Chapter \ref{sec::BF} has a direct 3D analogue. However, careful consideration is required to remove all of the diameter nodes within the polyhedron and is an open area of research in the applied mathematics community.
\item {\em Higher-order 2D serendipity polygonal basis functions:} \\
The quadratic serendipity basis functions were formed by taking pairwise products of the linear barycentric basis functions, followed by removal of the interior nodes. For a given polynomial order $p$, the monomials that the basis functions need to exactly interpolate are $x^\sigma y^\tau$, where $\sigma + \tau = p$. We can see that all of the higher-order functional spaces can be generated by taking pairwise products of terms from lower-order functions. Mukherjee and Webb have just recently developed a means to generate these higher-order polygonal finite elements through a hierarchical approach \cite{mukherjee2015hierarchical}.
\item {\em Alternate integration schemes on polygons:} \\
For this work, our quadrature integration scheme on arbitrary polygons consisted of a simple triangulation scheme where each sub-triangle had points mapped onto it from the reference triangle. We did not focus on efficiency for this work, but instead simply used a high-order reference quadrature set. However, by performing our integration this way, the basis function values and gradients must be computed for each polygon in the mesh. This becomes computationally expensive for meshes with many cells containing polygons with large vertex counts. An alternative approach could consist of the use of Schwarz-Christoffel Conforming Maps (SCCM) \cite{driscoll2002schwarz,driscoll2005algorithm}. Generation of the polygonal basis functions and gradients could be computed on reference (regular) polygons and then conformally mapped to any arbitrary polygon for integration \cite{natarajan2009numerical}.
\item {\em Mixed-mode parallelism with DSA preconditioning:} \\
In this work, spatial parallelism was done with domain decomposition using the Message Passing Interface (MPI).  
\item {\em Further preconditioning the MJA method:} \\
The Multigroup Jacobi Acceleration (MJA) method that we presented in Chapter \ref{sec::DSA} does not require the serialization of the energy group transport solves like the TG and MTG methods. This makes it more amenable to parallelizable computational transport methods. However, since the inner iterations are not converged with MJA, we see degradation of the iterative spectral radius compared to TG (~0.95 compared to ~0.4 for pure graphite). 
	\begin{enumerate}
	\item Perform a single 1 group set transport sweep for all thermal groups.
	\item For each thermal group, perform a 1-group DSA acceleration step for the within-group, 0th moment scattering residual. This correspdonds to the following error equations ($g=1,...,G$):
	        \begin{equation*}
	        -\vec{\nabla} \cdot D_g \vec{\nabla} \delta \phi_g^{(k+1/2)} + \left( \sigma_{t,g} - \sigma_{s,0}^{gg} \right) \delta \phi_g^{(k+1/2)} = \sigma_{s,0}^{gg} \left( \phi_g^{(k+1/2)} - \phi_g^{(k)}\right)
	        \end{equation*}
	\item Perform a single 1-group, energy-collapsed 
	\end{enumerate}
\end{enumerate}