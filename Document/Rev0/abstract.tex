%%%%%%%%%%%%%%%%%%%%%%%%%%%%%%%%%%%%%%%%%%%%%%%%%%%
%
%  New template code for TAMU Theses and Dissertations starting Fall 2012.  
%  For more info about this template or the 
%  TAMU LaTeX User's Group, see http://www.howdy.me/.
%
%  Author: Wendy Lynn Turner 
%	 Version 1.0 
%  Last updated 8/5/2012
%
%%%%%%%%%%%%%%%%%%%%%%%%%%%%%%%%%%%%%%%%%%%%%%%%%%%
%%%%%%%%%%%%%%%%%%%%%%%%%%%%%%%%%%%%%%%%%%%%%%%%%%%%%%%%%%%%%%%%%%%%%
%%                           ABSTRACT 
%%%%%%%%%%%%%%%%%%%%%%%%%%%%%%%%%%%%%%%%%%%%%%%%%%%%%%%%%%%%%%%%%%%%%

\chapter*{ABSTRACT}
\addcontentsline{toc}{chapter}{ABSTRACT} % Needs to be set to part, so the TOC doesnt add 'CHAPTER ' prefix in the TOC.

\pagestyle{plain} % No headers, just page numbers
\pagenumbering{roman} % Roman numerals
\setcounter{page}{2}

\indent In this dissertation, we develop improvements to the discrete ordinates ($S_N$) neutron transport equation using a Discontinuous Galerkin Finite Element Method (DGFEM) spatial discretization on arbitrary polytope (polygonal and polyhedral) grids compatible for massively-parallel computer architectures. Polytope meshes are attractive for multiple reasons, including their use in other physics communities and their ease in handling local mesh refinement strategies. In this work, we focus on two topical areas of research. First, we discuss higher-order basis functions compatible to solve the DGFEM $S_N$ transport equation on arbitrary polygonal meshes. Second, we assess Diffusion Synthetic Acceleration (DSA) schemes compatible with polytope grids for massively-parallel transport problems.

We first utilize basis functions compatible with arbitrary polygonal grids for the DGFEM transport equation. We analyze four different basis functions that have linear completeness on polygons: the Wachspress rational functions, the PWL functions, the mean value coordinates, and the maximum entropy coordinates. We then describe the procedure to extend these polygonal linear basis functions into the quadratic serendipity space of functions. These quadratic basis functions can exactly interpolate monomial functions up to order 2. Both the linear and quadratic sets of basis functions preserve transport solutions in the thick diffusion limit. Maximum convergence rates of 2 and 3 are observed for regular transport solutions for the linear and quadratic basis functions, respectively. For problems that are limited by the regularity of the transport solution, convergence rates of 3/2 (when the solution is continuous) and 1/2 (when the solution is discontinuous) are observed. Spatial Adaptive Mesh Refinement (AMR) achieved superior convergence rates than uniform refinement, even for problems bounded by the solution regularity. We demonstrated accuracy in the AMR solutions by allowing them to reach a level where the ray effects of the angular discretization are realized.

Next, we analyzed DSA schemes using a DGFEM discretization of the diffusion equation that is compatible with arbitrary polytope meshes: the Modified Interior Penalty Method (MIP). MIP uses the same DGFEM discretization as the transport equation. The MIP form is Symmetric Positive Definite (SPD) and efficiently solved with Preconditioned Conjugate Gradient (PCG) with Algebraic MultiGrid (AMG) preconditioning. The analysis from previous work was extended to show MIP's stability and robustness for accelerating 3D transport problems. MIP DSA preconditioning was implemented in the Parallel Deterministic Transport (PDT) code at Texas A\&M University and linked with the HYPRE suite of linear solvers. Good scalability was numerically verified out to around 131K processors. The fraction of time spent performing DSA operations was small for problems with sufficient work performed in the transport sweep ($O(10^3)$ angular directions). Finally, we have developed a novel methodology to accelerate transport problems dominated by thermal neutron upscattering. Compared to historical upscatter acceleration methods, our method is parallelizable and amenable to massively parallel transport calculations. Speedup factors of about 3-4 were observed with our new method.

\pagebreak{}
