%%%%%%%%%%%%%%%%%%%%%%%%%%%%%%%%%%%%%%%%%%%%%%%%%%%
%
%  New template code for TAMU Theses and Dissertations starting Fall 2012.  
%  For more info about this template or the 
%  TAMU LaTeX User's Group, see http://www.howdy.me/.
%
%  Author: Wendy Lynn Turner 
%	 Version 1.0 
%  Last updated 8/5/2012
%
%%%%%%%%%%%%%%%%%%%%%%%%%%%%%%%%%%%%%%%%%%%%%%%%%%%
%%                           SECTION  - Introduction
%%%%%%%%%%%%%%%%%%%%%%%%%%%%%%%%%%%%%%%%%%%%%%%%%%%
\pagestyle{plain} % No headers, just page numbers
\pagenumbering{arabic} % Arabic numerals
\setcounter{page}{1}

%%%%%%%%%%%%%%%%%%%%%%%%%%%%%%%%%%%%%%%%%%%%%%%%%%%
\chapter{\uppercase {Introduction}}
\label{sec::Intro}


%%%%%%%%%%%%%%%%%%%%%%%%%%%%%%%%%%%%%%%%%%%%%%%%%%%
%%%   Section - Past Work
\section{Current State of the Problem}
\label{sec::Intro_Past}


%%%%%%%%%%%%%%%%%%%%%%%%%%%%%%%%%%%%%%%%%%%%%%%%%%%
%%%   Section - Purpose
\section{Motivation and Purpose of the Dissertation}
\label{sec::Intro_Purpose}


%%%%%%%%%%%%%%%%%%%%%%%%%%%%%%%%%%%%%%%%%%%%%%%%%%%
%%%   Section - Organization
\section{Organization of the Dissertation}
\label{sec::Intro_Organization}

In this introductory chapter, we have presented a summary of work performed. We also gave our motivation for choosing this work as well as a brief discussion of previous work that has directly influenced this dissertation. We conclude this intoduction by briefly describing the remaining chapters of this dissertation.

In Chapter \ref{sec::Sn}, we present the DGFEM formulation for the multigroup, $S_N$ transport equation. We then describe the transport equation's discretization in energy, angle, and space. We have left the FEM spatial interpolation function as arbitrary at this point to be defined in detail in Chapter \ref{sec::BF}. For the spatial variable, we provide the theoretical convergence properties of the DGFEM form. We also detail the elementary assembly procedures to form the full set of spatial equations. We conclude by providing the methodology to be used to solve the full phase-space of the transport problem.

In Chapter \ref{sec::BF}, we present all the finite element basis functions that we will use in this work. In two dimensions, we present four different linearly-complete polygonal coordinate systems that we will use to generate our finite element basis functions. We then present the methodology that converts each of these linear coordinate systems into quadratically-complete coordinates for use as higher-order basis functions. We also present the 

In Chapter \ref{sec::DSA}, we present 

We then finalize this dissertation work by drawing conclusions and discussing open topics of research stemming from this dissertation in Chapter \ref{sec::Conclusions}. We note that our detailed literature reviews, numerical results, and conclusions pertaining to each topic are presented in their corresponding chapter.

Additional material that is not included in the main body of the dissertation for the sake of brevity is appended for completeness. The appendices are organized in a simple manner:

\begin{itemize}
\item Appendix \ref{sec::appendix_SN}: addendum to Section \ref{sec::Sn}, corresponding to additional material relating to the multigroup $S_N$ equations.
\item Appendix \ref{sec::appendix_BF}: addendum to Section \ref{sec::BF}, corresponding to additional material relating to the various polytope coordinate systems to be utilized as finite element basis functions.
\item Appendix \ref{sec::appendix_DSA}: addendum to Section \ref{sec::DSA}, corresponding to additional material relating to DSA preconditioning on polytope grids.
\end{itemize}