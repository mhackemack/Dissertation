%%%%%%%%%%%%%%%%%%%%%%%%%%%%%%%%%%%%%%%%%%%%%%%%%%%
%
%  New template code for TAMU Theses and Dissertations starting Fall 2012.  
%  For more info about this template or the 
%  TAMU LaTeX User's Group, see http://www.howdy.me/.
%
%  Author: Wendy Lynn Turner 
%	 Version 1.0 
%  Last updated 8/5/2012
%
%%%%%%%%%%%%%%%%%%%%%%%%%%%%%%%%%%%%%%%%%%%%%%%%%%%
%%                           SECTION  - Introduction
%%%%%%%%%%%%%%%%%%%%%%%%%%%%%%%%%%%%%%%%%%%%%%%%%%%
\pagestyle{plain} % No headers, just page numbers
\pagenumbering{arabic} % Arabic numerals
\setcounter{page}{1}

%%%%%%%%%%%%%%%%%%%%%%%%%%%%%%%%%%%%%%%%%%%%%%%%%%%
\chapter{\uppercase {Introduction}}
\label{sec::Intro}

%%%%%%%%%%%%%%%%%%%%%%%%%%%%%%%%%%%%%%%%%%%%%%%%%%%
%%%   Section - Purpose
\section{Motivation and Purpose of the Dissertation}
\label{sec::Intro_Purpose}


\begin{enumerate}
	\item Polytope mesh cells are now being employed in other physics communities - most notably computational fluid dynamics (CFD) \cite{ref::star_CCM};
	\item They are believed to reduce the number of unknowns to solve with equivalent accuracy;
	\item They can reduce cell/face counts which can reduce algorithm wallclock times depending on the solution method;
	\item They can allow for transition elements between different portions of the domain (e.g., tetrahedral elements bordering hexahedral elements at the border of the boundary layer);
	\item They can easily be split along cut planes - allowing the mesh to be partitioned into regular or irregular divisions as well as be generated by simplical meshing techniques across processor sets in parallel;
	\item Hanging nodes from non-conforming meshes, like those that naturally arise from locally refined/adapted meshes, are no longer necessary. 
\end{enumerate}


%%%%%%%%%%%%%%%%%%%%%%%%%%%%%%%%%%%%%%%%%%%%%%%%%%%
%%%   Section - Current State of the Problem
\section{Current State of the Problem}
\label{sec::Intro_Past}

%%%%%%%%%%%%%%%%%%%%%%%%%%%%%%%%%%%%%%%%%%%%%%%%%%%
%%%   SubSection - DGFEM Sn Transport
\subsection{Background on the Multigroup DGFEM $S_N$ Transport Equation}
\label{sec::Intro_Past_DGFEMMGSn}

%%%%%%%%%%%%%%%%%%%%%%%%%%%%%%%%%%%%%%%%%%%%%%%%%%%
%%%   SubSection - DSA
\subsection{Diffusion Synthetic Acceleration}
\label{sec::Intro_Past_DSA}

%%%%%%%%%%%%%%%%%%%%%%%%%%%%%%%%%%%%%%%%%%%%%%%%%%%
%%%   SubSection - Voronoi
\subsection{Polytope Grids Formed from Voronoi Mesh Generation}
\label{sec::Intro_Past_Voronoi}

Since this dissertation work is on the solution of the transport equation on polytope meshes, we next describe how these grids can be generated. Traditionally, FEM calculations have been performed on simplical (triangles and tetrahedra) and tensor based meshes (quadrilaterals and hexahedra). In fact, it is still a standard practice to refer to any type of mesh as a {\em triangulation} in some communities \cite{ern2013theory}. Many different mesh generation software has been developed to build these simple grids \cite{shewchuk1996triangle,shewchuk2002delaunay,si2015tetgen,geuzaine2009gmsh}. However, multiple fields including {\em computational fluid dynamics} (CFD) and {\em solid mechanics} are now finding benefits to utilizing polygonal and polyhedral meshes for their calculations.

However, polytope mesh generation is still in its infancy. 

%%%%%%%%%%%%%%%%%%%%%%%%%%%%%%%%%%%%%%%%%%%%%%%%%%%
%%%   SubSection - AMR
\subsection{Adaptive Mesh Refinement for the DGFEM $S_N$ Transport Equation}
\label{sec::Intro_Past_AMR}

%%%%%%%%%%%%%%%%%%%%%%%%%%%%%%%%%%%%%%%%%%%%%%%%%%%
%%%   Section - Organization
\section{Organization of the Dissertation}
\label{sec::Intro_Organization}

In this introductory chapter, we have presented a summary of work performed. We also gave our motivation for choosing this work as well as a brief discussion of previous work that has directly influenced this dissertation. We conclude this intoduction by briefly describing the remaining chapters of this dissertation.

In Chapter \ref{sec::Sn}, we present the DGFEM formulation for the multigroup, $S_N$ transport equation. We then describe the transport equation's discretization in energy, angle, and space. We have left the FEM spatial interpolation function as arbitrary at this point to be defined in detail in Chapter \ref{sec::BF}. For the spatial variable, we provide the theoretical convergence properties of the DGFEM form. We also detail the elementary assembly procedures to form the full set of spatial equations. We conclude by providing the methodology to be used to solve the full phase-space of the transport problem.

In Chapter \ref{sec::BF}, we present all the finite element basis functions that we will use in this work. In two dimensions, we present four different linearly-complete polygonal coordinate systems that we will use to generate our finite element basis functions. We then present the methodology that converts each of these linear coordinate systems into quadratically-complete coordinates for use as higher-order basis functions. We also present the single linearly-complete polyhedral coordinate system that we will use for the 3D transport problems.

In Chapter \ref{sec::DSA}, we present the methodologies to be used for DSA preconditioning of the DGFEM transport equation for optically thick problems. We give a discontinuous form of the diffusion equation which can be used on 2D and 3D polytope grids. The theoretical limits of the DSA scheme are analyzed and we conclude with a real-world problem of accelerating the thermal neutron upscattering of a large multigroup, heterogeneous transport problem. In doing so, we demonstrate that our methodology will work on massively-parallel computer architectures.

We then finalize this dissertation work by drawing conclusions and discussing open topics of research stemming from this dissertation in Chapter \ref{sec::Conclusions}. We note that our detailed literature reviews, numerical results, and conclusions pertaining to each topic are presented in their corresponding chapter.

Additional material that is not included in the main body of the dissertation for the sake of brevity is appended for completeness. The appendices are organized in a simple manner:

\begin{itemize}
\item Appendix \ref{sec::appendix_SN}: addendum to Section \ref{sec::Sn}, corresponding to additional material relating to the multigroup $S_N$ equations.
\item Appendix \ref{sec::appendix_BF}: addendum to Section \ref{sec::BF}, corresponding to additional material relating to the various polytope coordinate systems to be utilized as finite element basis functions.
\item Appendix \ref{sec::appendix_DSA}: addendum to Section \ref{sec::DSA}, corresponding to additional material relating to DSA preconditioning on polytope grids.
\end{itemize}